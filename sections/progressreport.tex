\section{Εισαγωγή}
% Οι σύγχρονες εφαρμογές παράγουν τεράστιους όγκους χωροχρονικών δεδομένων διαδρομών καθημερινά, ποσότητα η οποία θα αυξάνεται όλο και περισσότερο με την πάροδο του χρόνου. Αυτά τα σύνολα δεδομένων, μετά από την κατάλληλη επεξεργασία, είναι κρίσιμα για την κατανόηση μοτίβων κινητικότητας, τα οποία επιτρέπουν την εξαγωγή συμπερασμάτων \cite{10.5555/2602439}, αλλά και προβλέψεων σημαντικής βαρύτητας για πολλαπλούς τομείς και ποικίλες εφαρμογές \cite{VourosBigDataAnalytics}, όπως η παρακολούθηση πλοίων/στόλων και αεροσκαφών, η αξιολόγηση επικινδυνότητας διαδρομών, η διαχείριση κυκλοφορίας οχημάτων (\latintext{traffic flow analytics}), οι εφαρμογές διαμοιρασμού διαδρομών (ride-sharing) κ.α. 
Η έλευση IoT συσκευών, όπως αισθητήρες, smartwatches, smartphones και GPS trackers, έχει οδηγήσει στην παραγωγή τεράστιου όγκου δεδομένων, συμπεριλαμβανομένων δεδομένων κινητικότητας, σε καθημερινή βάση, ποσότητα η οποία θα αυξάνεται όλο και περισσότερο με την πάροδο του χρόνου. Αυτά τα σύνολα δεδομένων, μετά από την κατάλληλη επεξεργασία, είναι κρίσιμα για την κατανόηση μοτίβων κινητικότητας \cite{DBLP:conf/edbt/TritsarolisCTP21,DBLP:journals/geoinformatica/TritsarolisCTPT24}, τα οποία επιτρέπουν την εξαγωγή συμπερασμάτων αλλά και προβλέψεων σημαντικής βαρύτητας για πολλαπλούς τομείς \cite{VourosBigDataAnalytics}, όπως η παρακολούθηση πλοίων/στόλων \cite{DBLP:conf/mdm/TritsarolisPBZT24}, η αξιολόγηση επικινδυνότητας διαδρομών \cite{DBLP:conf/gis/MichalskiPRTTP24}, η πρόβλεψη ροής κίνησης (\latintext{traffic flow prediction}) \cite{DBLP:conf/edbt/MandalisCKPT23}, οι εφαρμογές διαμοιρασμού διαδρομών (ride-sharing), κ.α. 

% Αυτός ο όγκος πληροφορίας καθιστά απαραίτητη την δημιουργία ενός workflow το οποίο θα είναι ικανό να γεφυρώσει τα ωμά γεωγραφικά δεδομένα και την ανθρώπινη κατανόηση ως προς τα αποτελέσματα που μπορεί να προσφέρει μια ανάλυση τους. Σε αυτή την ανάγκη πατάει ο ερευνητικός τομέας των Visual Analytics (VA), δημιουργώντας ένα πλαίσιο στο οποίο μέθοδοι και λογισμικά εκμεταλλεύονται την ανθρώπινη "όραση" και λήψη αποφάσεων για την εξόρυξη γνώσης από χωροχρονικά δεδομένα \cite{vision2020}, όχι μόνο κινητικότητας (Mobility Data) \cite{MaSEC} άλλα και άλλου χαρακτήρα (Κοινωνικά Δίκτυα, Οικονομία, Κλιματική Επιστήμη) \cite{SensePlace3,EconomicVA,nayeem2024}.
Αυτός ο όγκος πληροφορίας καθιστά απαραίτητη την δημιουργία ενός workflow το οποίο θα είναι ικανό να γεφυρώσει την ``καθαρή'' γεωγραφική πληροφορία με την ανθρώπινη κατανόηση ως προς τα αποτελέσματα που μπορεί να προσφέρει μια ανάλυσή τους. Σε αυτή την ανάγκη πατάει ο ερευνητικός τομέας των Visual Analytics (VA), δημιουργώντας ένα πλαίσιο στο οποίο μέθοδος και λογισμικό εκμεταλλεύονται την ανθρώπινη ``όραση'' και λήψη αποφάσεων για την εξόρυξη γνώσης από χωροχρονικά δεδομένα \cite{vision2020}, τόσο κίνησης (Mobility Data) \cite{MaSEC,DBLP:conf/gis/TritsarolisPT23}, όσο και άλλου χαρακτήρα (Κοινωνικά Δίκτυα, Οικονομία, Κλιματική Επιστήμη) \cite{SensePlace3,EconomicVA,nayeem2024}.